\documentclass[12pt]{article}
\usepackage[T2A]{fontenc}
\usepackage[utf8]{inputenc}
\usepackage[russian]{babel}
\usepackage{amsmath}
\usepackage{amsthm}
\usepackage{amssymb}

\begin{document}
	Антон Каразеев, 493\\
	
	\textbf{3. Теоретические задачи.}
	
	\textbf{3.1 Знакомство с линейным классификатором}
	
	\begin{enumerate}
		\item Как выглядит бинарный линейный классификатор?
			
			Есть два класса объектов A=\{-1, +1\}. Отображение $f(x): X \rightarrow A$ называется классификатором, отображающим объекты из множества X во множество классов A. Линейный классификатор выглядит следующим образом: $f(x) = sign(w^Tx + w_0)$.
		
		\item Что такое отступ алгоритма на объекте? Какие выводы можно сделать из знака отступа?
			
			В общем виде отступ $M(x_i) = y_i g(x_i)$, где $y_i$ - метка $i$-того класса. Так как множество классов A=\{-1, +1\}, то можно сделать вывод о том, что при правильном отнесении объекта к классу $M(x_i)$ положителен. В противном случае - отрицательный. Следовательно, неположительный отступ - ошибка классификатора.
			
		\item Как классификаторы вида $a(x)=sign(<w,x> - w_0)$ сводят к классификаторам вида $a(x)=sign(<w,x>)$?
			
			К вектору $x$ добавляют еще одну координату со значением $-1$, а к вектору $w$ --- $w_0$.
			
		\item Как выглядит запись функционала эмпирического риска через отступы? Какое значение он должен принимать для "наилучшего" алгоритма классификации?
		
			$Q(X) = \sum_{x \in X} I\{M(x) < 0\}$
			
			Для "наилучшего" алгоритма классификации он должен принимать значение 0.
			
		\item Если в функционале эмпирического риска (риск с пороговой функцией потерь) всюду написаны строгие неравенства ($M_i < 0$) можете ли вы сразу  придумать параметр $w$ для алгоритма классификации $a(x) = sign(<w, x>)$, минимизирующий такой функционал?
		
			Положить $w=0$.
		
		\item Запишите функционал аппроксимированного эмпирического риска, если выбрана функция потерь $L(M)$.
		
		
		
		
	\end{enumerate}
\end{document}