\documentclass[12pt]{article}
\usepackage[T2A]{fontenc}
\usepackage[utf8]{inputenc}
\usepackage[russian]{babel}
\usepackage{amsmath}
\usepackage{amsthm}
\usepackage{amssymb}

\begin{document}
	Антон Каразеев, 493\\
	
	\textbf{3. Теоретические задачи.}
	
	\textbf{3.1 Знакомство с линейным классификатором}
	
	\begin{enumerate}
		\item Как выглядит бинарный линейный классификатор?
			
			Есть два класса объектов A=\{-1, +1\}. Отображение $f(x): X \rightarrow A$ называется классификатором, отображающим объекты из множества X во множество классов A.
		
		\item Что такое отступ алгоритма на объекте? Какие выводы можно сделать из знака отступа?
	\end{enumerate}
\end{document}